\chapter{重温傅里叶级数}
    在第一章中,我们从傅里叶级数开始推导出傅里叶积分公式以及傅里叶变换。但是,这样的推导在数学上是不完整的。下面试图从另一个更加基础的方面来诠释傅里叶级数理论。
    \section{从周期性开始}
        傅里叶级数源于傅里叶的一个简单的想法:\emph{任何}函数都可以表示成一组正弦函数的有限和。当然,从数学分析的角度来说,这样的说法是相当不严谨的,但我们不%
        妨暂时承认这个假设,看看能够走到多远。
        假设一个函数$f(t)$可以表示成一串正弦函数的和,那么
        \begin{equation*}
            f(t) = \sum_{n = 1}^{N} A_n\sin(2\pi n t + \varphi_n)
        \end{equation*}
        但是这个式子很不好看,虽然三个变量可以唯一地决定一个正弦函数,但是在计算的时候,将会遇到很多的障碍(正弦内有加减)。所以,通常将上式变换一下:
        \begin{equation}
            \begin{split}
                A_n\sin(2\pi n t + \varphi_n) &= A_n(\sin(2\pi nt)\cos\varphi_n + \cos(2\pi nt)\sin\varphi_n)\\
                &= (A_n\cos\varphi_n)\sin(2\pi nt) + (A_n\sin\varphi_n)\cos(2\pi nt)\\
                &= a_n\cos(2\pi nt) + b_n\sin(2\pi nt)
            \end{split}
            \label{eq: 3.1}
        \end{equation}
        这里按照习惯,将$\cos(2\pi nt)$放在前面,是无关紧要的。

        但上面的式子有一个不能表示的函数,那就是常函数,所以,可以在这个级数的前面加上一个常数项。至于这个常数项到底是多少,其实是很难判断的。由于$n = 0$时,%
        \myref{eq: 3.1}的结果是$a_0$,所以暂且将常数项定为$ka_0$($k$为系数),于是:
        \begin{equation}
            f(t) = ka_0 + \sum_{n = 1}^{N} a_n\cos(2\pi nt) + b_n\sin(2\pi nt)
            \label{eq: 3.2}
        \end{equation}
        正如我们在第一章中推导过的,通过欧拉公式的变换,可以将上面的这个级数变成
        \begin{equation*}
            \begin{split}
                f(t) &= ka_0 + \sum_{n = 1}^{N} a_n\cos(2\pi nt) + b_n\sin(2\pi nt)\\
                &= ka_0 + \sum_{n = 1}^{N} a_n\dfrac{e^{i2\pi nt} + e^{-i2\pi nt}}{2} + b_n\dfrac{e^{i2\pi nt} - e^{-i2\pi nt}}{2i}\\
                &= ka_0 + \sum_{n = 1}^{N} \frac 1 2 (a_n - ib_n)e^{i2\pi nt} + \frac 1 2 (a_n + ib_n) e^{-i2\pi nt}\\
                &= ka_0 + \sum_{n = 1}^{N} c_n e^{i2\pi nt} + c_{-n} e^{-i2\pi nt}
            \end{split}
        \end{equation*}
        从上面的推导可以看出,如果$k = \frac 1 2$的话,那将是比较合理的,可以将所有的$c_n$合并起来
        \begin{equation}
            f(t) = \sum_{n = -N}^{N} c_n e^{i2\pi nt}
            \label{eq: 3.3}
        \end{equation}
        注意到,$c_n = \overline{c_{-n}}$,$e^{i2\pi nt} = \overline{e^{-i2\pi nt}}$,所以,这个级数是关于实轴对称的。由于共轭复数的和是两倍的实部,%
        那么,\myref{eq: 3.3}就应该是一个实函数:
        \begin{equation*}
            \sum_{n = -N}^{N} c_n e^{i2\pi nt} = c_0 + 2Re\left\{\sum_{n = 1}^{N} c_n e^{i2\pi nt}\right\}
        \end{equation*}
        这样就大可放心了,$f(t)$是一个实变函数,经过一个复数的变换还是一个实变函数,不会跑到复数域去。

        \myref{eq: 3.3}的结论是一个优美的结论,但是这一切都源于对一个假设的默认:任何函数都可以表示成一组正弦函数的有限和。既然我们通过这个假设得到%
        了\myref{eq: 3.3},那就要对这个假设负责,解决下面两个问题:

        \begin{enumerate}
            \item $c_k$的值如何构造,是什么?
            \item $\sum\limits_{n = -N}^{N} c_n e^{i2\pi nt}$真的可以趋近$f(t)$吗?如果可以,它趋近的速度如何?
        \end{enumerate}
        在下一节中,我们将重点解决这两个问题。